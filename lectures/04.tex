\documentclass{article}

\usepackage{amssymb, amsmath}
\usepackage{alltt}
\usepackage{pslatex}
\usepackage{epigraph}
\usepackage{verbatim}
\usepackage{latexsym}
\usepackage{array}
\usepackage{comment}
\usepackage{makeidx}
\usepackage{listings}
\usepackage{indentfirst}
\usepackage{verbatim}
\usepackage{color}
\usepackage{url}
\usepackage{xspace}
\usepackage{hyperref}
\usepackage{stmaryrd}
\usepackage{amsmath, amsthm, amssymb}
\usepackage{graphicx}
\usepackage{euscript}
\usepackage{mathtools}
\usepackage{mathrsfs}
\usepackage{multirow,bigdelim}
\usepackage{subcaption}
\usepackage{placeins}

\makeatletter

\makeatother

\definecolor{shadecolor}{gray}{1.00}
\definecolor{darkgray}{gray}{0.30}

\def\transarrow{\xrightarrow}
\newcommand{\setarrow}[1]{\def\transarrow{#1}}

\def\padding{\phantom{X}}
\newcommand{\setpadding}[1]{\def\padding{#1}}

\def\subarrow{}
\newcommand{\setsubarrow}[1]{\def\subarrow{#1}}

\newcommand{\trule}[2]{\frac{#1}{#2}}
\newcommand{\crule}[3]{\frac{#1}{#2},\;{#3}}
\newcommand{\withenv}[2]{{#1}\vdash{#2}}
\newcommand{\trans}[3]{{#1}\transarrow{\padding{\textstyle #2}\padding}\subarrow{#3}}
\newcommand{\ctrans}[4]{{#1}\transarrow{\padding#2\padding}\subarrow{#3},\;{#4}}
\newcommand{\llang}[1]{\mbox{\lstinline[mathescape]|#1|}}
\newcommand{\pair}[2]{\inbr{{#1}\mid{#2}}}
\newcommand{\inbr}[1]{\left<{#1}\right>}
\newcommand{\highlight}[1]{\color{red}{#1}}
\newcommand{\ruleno}[1]{\eqno[\scriptsize\textsc{#1}]}
\newcommand{\rulename}[1]{\textsc{#1}}
\newcommand{\inmath}[1]{\mbox{$#1$}}
\newcommand{\lfp}[1]{fix_{#1}}
\newcommand{\gfp}[1]{Fix_{#1}}
\newcommand{\vsep}{\vspace{-2mm}}
\newcommand{\supp}[1]{\scriptsize{#1}}
\newcommand{\sembr}[1]{\llbracket{#1}\rrbracket}
\newcommand{\cd}[1]{\texttt{#1}}
\newcommand{\free}[1]{\boxed{#1}}
\newcommand{\binds}{\;\mapsto\;}
\newcommand{\dbi}[1]{\mbox{\bf{#1}}}
\newcommand{\sv}[1]{\mbox{\textbf{#1}}}
\newcommand{\bnd}[2]{{#1}\mkern-9mu\binds\mkern-9mu{#2}}
\newtheorem{lemma}{Lemma}
\newtheorem{theorem}{Theorem}
\newcommand{\meta}[1]{{\mathcal{#1}}}
\renewcommand{\emptyset}{\varnothing}
\newcommand{\dom}[1]{\mathtt{dom}\;{#1}}
\newcommand{\primi}[2]{\mathbf{#1}\;{#2}}

\definecolor{light-gray}{gray}{0.90}
\newcommand{\graybox}[1]{\colorbox{light-gray}{#1}}

\lstdefinelanguage{lama}{
keywords={read, write, skip,if,then,else,elif,fi,while,do,od,repeat,until,for,fun,local,public,return,import,length,
string,case,of,esac,when,boxed,unboxed,string,sexp,array,infix,infixl,infixr,at,before,after,true,false,eta,lazy},
sensitive=true,
basicstyle=\small,
%commentstyle=\scriptsize\rmfamily,
keywordstyle=\ttfamily\bfseries,
identifierstyle=\ttfamily,
basewidth={0.5em,0.5em},
columns=fixed,
fontadjust=true,
literate={->}{{$\to$}}3,
morecomment=[s][\ttfamily]{(*}{*)},
morecomment=[l][\ttfamily]{--}
}

\lstset{
mathescape=true,
basicstyle=\small,
identifierstyle=\ttfamily,
keywordstyle=\bfseries,
commentstyle=\scriptsize\rmfamily,
basewidth={0.5em,0.5em},
fontadjust=true,
escapechar=!,
language=lama
}

\sloppy

\newcommand{\lama}{$\lambda\kern -.1667em\lower -.5ex\hbox{$a$}\kern -.1000em\lower .2ex\hbox{$\mathcal M$}\kern -.1000em\lower -.5ex\hbox{$a$}$\xspace}

\theoremstyle{definition}

\author{Dmitry Boulytchev}

\begin{document}

\section{Extended Stack Machine}

In order to compile a language with structural control flow constructs into a program for the stack machine the latter has to be extended. First, we introduce a set of label names

\[
\mathscr L = \{l_1, l_2, \dots\}
\]

Then, we add three extra control flow instructions:

\[
\begin{array}{rcl}
  \mathscr I & += & \llang{LABEL $\;\mathscr L$} \\
             &   & \llang{JMP $\;\mathscr L$} \\
  &   & \llang{CJMP$_x$ $\;\mathscr L$},\;\mbox{where $x\in\{\llang{nz}, \llang{z}\}$}
\end{array}
\]

In order to give the semantics to these instructions, we need to extend the syntactic form of rules, used in the description of big-step operational smeantics. Instead of
the rules in the form

\setarrow{\xRightarrow}
\setsubarrow{_{\mathscr{SM}}}

\[
\trule{\trans{c}{p}{c^\prime}}{\trans{c^\prime}{p^\prime}{c^{\prime\prime}}}
\]

we use the following form

\[
\trule{\withenv{\Gamma^\prime}{\trans{c}{p}{c^\prime}}}{\withenv{\Gamma}{\trans{c^\prime}{p^\prime}{c^{\prime\prime}}}}
\]

where $\Gamma, \Gamma^\prime$~--- \emph{environments}. The structure of environments can be different in different cases; for now environment is just a program. Informally,
the semantics of control flow instructions can not be described in terms of just a current instruction and current configuration~--- we need to take the whole
program into account. Thus, the enclosing program is used as an environment.

Additionally, for a program $P$ and a label $l$ we define a subprogram $P[l]$, such that $P$ is uniquely represented as $p^\prime[\llang{LABEL $\;l$}]P[l]$.
In other words $P[l]$ is a unique suffix of $P$, immediately following the label $l$ (if there are multiple (or no) occurrences of label $l$ in $P$, then $P[l]$ is
undefined).

All existing rules have to be rewritten~--- we need to formally add a $\withenv{P}{\dots}$ part everywhere. For the new instructions the rules are given on Fig.~\ref{bs_sm_cc}.

\begin{figure}[t]
  
  \renewcommand{\arraystretch}{2}

  \[\trule{\withenv{P}{\trans{c}{p}{c^\prime}}}{\withenv{P}{\trans{c}{[\llang{LABEL $\;l$}]p}{c^\prime}}}\ruleno{Label$_{SM}$}\]
  \[\trule{\withenv{P}{\trans{c}{P[l]}{c^\prime}}}{\withenv{P}{\trans{c}{[\llang{JMP $\;l$}]p}{c^\prime}}}\ruleno{JMP$_{SM}$}\]
  \[\trule{z\ne 0,\quad\withenv{P}{\trans{\inbr{s,\,\theta}}{P[l]}{c^\prime}}}{\withenv{P}{\trans{\inbr{zs,\,\theta}}{[\llang{CJMP$_{nz}$ $\;l$}]p}{c^\prime}}}\ruleno{CJMP$_{nz}^+$$_{SM}$}\]
  \[\trule{z = 0,\quad\withenv{P}{\trans{\inbr{s,\,\theta}}{p}{c^\prime}}}{\withenv{P}{\trans{\inbr{zs,\,\theta}}{[\llang{CJMP$_{nz}$ $\;l$}]p}{c^\prime}}}\ruleno{CJMP$_{nz}^-$$_{SM}$}\]
  \[\trule{z = 0,\quad\withenv{P}{\trans{\inbr{s,\,\theta}}{P[l]}{c^\prime}}}{\withenv{P}{\trans{\inbr{zs,\,\theta}}{[\llang{CJMP$_z$ $\;l$}]p}{c^\prime}}}\ruleno{CJMP$_{z}^+$$_{SM}$}\]
  \[\trule{z \ne 0,\quad\withenv{P}{\trans{\inbr{s,\,\theta}}{p}{c^\prime}}}{\withenv{P}{\trans{\inbr{zs,\,\theta}}{[\llang{CJMP$_z$ $\;l$}]p}{c^\prime}}}\ruleno{CJMP$_{z}^-$$_{SM}$}\]  
  \caption{Big-step operational semantics for extended stack machine}
  \label{bs_sm_cc}
\end{figure}

Finally, the top-level semantics for the extended stack machine can be redefined as follows:

\[
\trule{\withenv{p}{\trans{\inbr{\epsilon,\,\inbr{\Lambda,\,\inbr{i,\,\epsilon}}}}{p}{\inbr{s, \inbr{\sigma,\,\omega}}}}}
      {\sembr{p}_{\mathscr{SM}}\;i=\primi{out}{\omega}}
\]

\section{A Compiler for the Stack Machine}

A compiler for the language with structural control flow into the stack machine can be given in the form of static semantics. Similarly to the big-step operational semantics, the
compiler also operates on environment. For now, the environment allows us to generate fresh labels. Thus, a compiler specification for statements has the shape

\[
\sembr{p}_{\mathscr{S}}^{comp}\,\Gamma=\inbr{c,\,\Gamma^\prime}
\]

where $p$ is a source program, $\Gamma$, $\Gamma^\prime$~--- some environments, $c$~--- generated program for the stack machine. As we can see, the environment changes during the
code generation, hence auxilliary semantic primitive $\sembr{\bullet}_{\mathscr{S}}^{comp}$. We need one primitive to operate on environments which allocates a number of fresh
labels and returns a new environment:

\[
\primi{labels}\,\Gamma
\]

The number of labels allocated is determined by context.

We give an example of compiler specification rule for the while-loop:

\[
\trule{\inbr{l_e,\,l_s,\,\Gamma^\prime}=\primi{labels}\,\Gamma,\quad\sembr{s}_{\mathscr{S}}^{comp}\,\Gamma^\prime=\inbr{c_s,\,\Gamma^{\prime\prime}}}
      {\begin{array}{rcrll}
          \sembr{\llang{while $\;e\;$ do $\;s\;$ od}}_{\mathscr{S}}^{comp}\,\Gamma & = &  \left<\right. & \llang{JMP $\;l_e$}&\\
                                                                    &   &        & \llang{LABEL $\;l_s$}&\\
                                                                    &   &        & c_s&\\
                                                                    &   &        & \llang{LABEL $\;l_e$}&\\
                                                                    &   &        & \sembr{e}^{comp}_{\mathscr{E}}&\\
                                                                    &   &        & \llang{CJMP$_{nz}$ $\;l_s$},&\Gamma^{\prime\prime}\left.\right>
       \end{array}}
\]

Note, the compiler for expressions is not changed and completely reused.

Finally, the top-level compiler for the whole program can be defined as follows:

\[
\trule{\sembr{p}_{\mathscr{S}}^{comp}\,\Gamma_0=\inbr{c,\,\_}}
      {\sembr{p}^{comp}=c}
\]

where $\Gamma_0$~--- empty environment.
\end{document}
