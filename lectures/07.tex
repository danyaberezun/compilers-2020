\documentclass{article}

\usepackage{amssymb, amsmath}
\usepackage{alltt}
\usepackage{pslatex}
\usepackage{epigraph}
\usepackage{verbatim}
\usepackage{latexsym}
\usepackage{array}
\usepackage{comment}
\usepackage{makeidx}
\usepackage{listings}
\usepackage{indentfirst}
\usepackage{verbatim}
\usepackage{color}
\usepackage{url}
\usepackage{xspace}
\usepackage{hyperref}
\usepackage{stmaryrd}
\usepackage{amsmath, amsthm, amssymb}
\usepackage{graphicx}
\usepackage{euscript}
\usepackage{mathtools}
\usepackage{mathrsfs}
\usepackage{multirow,bigdelim}
\usepackage{subcaption}
\usepackage{placeins}

\makeatletter

\makeatother

\definecolor{shadecolor}{gray}{1.00}
\definecolor{darkgray}{gray}{0.30}

\def\transarrow{\xrightarrow}
\newcommand{\setarrow}[1]{\def\transarrow{#1}}

\def\padding{\phantom{X}}
\newcommand{\setpadding}[1]{\def\padding{#1}}

\def\subarrow{}
\newcommand{\setsubarrow}[1]{\def\subarrow{#1}}

\newcommand{\trule}[2]{\dfrac{#1}{#2}}
\newcommand{\crule}[3]{\frac{#1}{#2},\;{#3}}
\newcommand{\withenv}[2]{{#1}\vdash{#2}}
\newcommand{\trans}[3]{{#1}\transarrow{\padding{\textstyle #2}\padding}\subarrow{#3}}
\newcommand{\ctrans}[4]{{#1}\transarrow{\padding#2\padding}\subarrow{#3},\;{#4}}
\newcommand{\llang}[1]{\mbox{\lstinline[mathescape]|#1|}}
\newcommand{\pair}[2]{\inbr{{#1}\mid{#2}}}
\newcommand{\inbr}[1]{\left<{#1}\right>}
\newcommand{\highlight}[1]{\color{red}{#1}}
\newcommand{\ruleno}[1]{\eqno[\scriptsize\textsc{#1}]}
\newcommand{\rulename}[1]{\textsc{#1}}
\newcommand{\inmath}[1]{\mbox{$#1$}}
\newcommand{\lfp}[1]{fix_{#1}}
\newcommand{\gfp}[1]{Fix_{#1}}
\newcommand{\vsep}{\vspace{-2mm}}
\newcommand{\supp}[1]{\scriptsize{#1}}
\newcommand{\sembr}[1]{\llbracket{#1}\rrbracket}
\newcommand{\cd}[1]{\texttt{#1}}
\newcommand{\free}[1]{\boxed{#1}}
\newcommand{\binds}{\;\mapsto\;}
\newcommand{\dbi}[1]{\mbox{\bf{#1}}}
\newcommand{\sv}[1]{\mbox{\textbf{#1}}}
\newcommand{\bnd}[2]{{#1}\mkern-9mu\binds\mkern-9mu{#2}}
\newtheorem{lemma}{Lemma}
\newtheorem{theorem}{Theorem}
\newcommand{\meta}[1]{{\mathcal{#1}}}
\renewcommand{\emptyset}{\varnothing}
\newcommand{\dom}[1]{\mathtt{dom}\;{#1}}
\newcommand{\primi}[2]{\mathbf{#1}\;{#2}}

\definecolor{light-gray}{gray}{0.90}
\newcommand{\graybox}[1]{\colorbox{light-gray}{#1}}

\lstdefinelanguage{lama}{
keywords={read, write, skip,if,then,else,elif,fi,while,do,od,repeat,until,for,fun,local,public,return,import,length,
string,case,of,esac,when,boxed,unboxed,string,sexp,array,infix,infixl,infixr,at,before,after,true,false,eta,lazy,ignore, ref},
sensitive=true,
basicstyle=\small,
%commentstyle=\scriptsize\rmfamily,
keywordstyle=\ttfamily\bfseries,
identifierstyle=\ttfamily,
basewidth={0.5em,0.5em},
columns=fixed,
fontadjust=true,
literate={->}{{$\to$}}3,
morecomment=[s][\ttfamily]{(*}{*)},
morecomment=[l][\ttfamily]{--}
}

\lstset{
mathescape=true,
basicstyle=\small,
identifierstyle=\ttfamily,
keywordstyle=\bfseries,
commentstyle=\scriptsize\rmfamily,
basewidth={0.5em,0.5em},
fontadjust=true,
escapechar=!,
language=lama
}

\sloppy

\newcommand{\lama}{$\lambda\kern -.1667em\lower -.5ex\hbox{$a$}\kern -.1000em\lower .2ex\hbox{$\mathcal M$}\kern -.1000em\lower -.5ex\hbox{$a$}$\xspace}

\theoremstyle{definition}

\author{Dmitry Boulytchev}

\begin{document}

\section{Functions and Local Scopes in Stack Machine}

To support functions and local scopes the stack machine has to be essentially redesigned.

First, we add a new notion~--- \emph{location} ($\mathcal Loc$)~--- to the definition of stack machine. A location specifies where a non-stack operand of an instruction
resides. For now the three kinds of locations are sufficient:

\[
\begin{array}{rl}
  \primi{global}{\mathscr X} & \mbox{--- global variable}\\
  \primi{local}{\mathbb N}   & \mbox{--- local variable}\\
  \primi{arg}{\mathbb N}     & \mbox{--- function argument}
\end{array}
\]

Thus, now operands for instructions \llang{ST}, \llang{LD} and \llang{LDA} are locations. Moreover, the set of \emph{values} for stack machine now contains
references to locations as well as plain integer numbers:

\[
\mathscr V = \mathbb Z \mid \primi{ref}{\mathcal Loc}
\]

Next, we need a whole new bunch of instructions:

\[
\begin{array}{rl}
  \llang{GLOBAL $\;\mathscr X$}                      & \mbox{--- declaration of global variable}\\
  \llang{CALL $\;\mathscr X\;\mathbb N$}             & \mbox{--- function call}\\
  \llang{BEGIN $\;\mathscr X\;\mathbb N\;\mathbb N$} & \mbox{--- begin of function}\\
  \llang{END}                                        & \mbox{--- end of function}
\end{array}
\]

Next to last, in addition to a regular state we add the notion of \emph{local state}:

\[
\Sigma_{loc} = (\mathbb N \to \mathscr V) \times (\mathbb N \to \mathscr V)
\]

Local states keep values of arguments and local variables, indexed by their numbers, respectively.

Finally, we modify the configuration for stack machine:

\[
\mathscr C = \mathscr V^* \times (\Sigma_{loc}\times\mathscr P)^* \times (\Sigma_{loc} \times \Sigma) \times \mathscr W
\]

In addition to a regular stack of values, global state and a world now the configurations contains two more
items:

\begin{itemize}
\item a \emph{control stack}, which is a stack of pairs of local state and programs, which keeps track of
  return points;
\item a local state, which keeps a current local state.
\end{itemize}

For extended state we need to refedine the primitives for reading

\[
\begin{array}{rlcl}
  \inbr{\inbr{a,\,l},\,g} &[\primi{local}{n}] & = & l\,(n)\\
  \inbr{\inbr{a,\,l},\,g} &[\primi{arg}{n}] & = & a\,(n)\\
  \inbr{\inbr{a,\,l},\,g} &[\primi{global}{x}] & = & g\,(x)
  \end{array}
\]

and the assignment

\[
\begin{array}{rlcl}
  \inbr{\inbr{a,\,l},\,g} &[\primi{local}{n}\gets v]  & = & \inbr{\inbr{a,\,l[i\gets v]},\,g}\\
  \inbr{\inbr{a,\,l},\,g} &[\primi{arg}{n}\gets v]    & = & \inbr{\inbr{a[i\gets v],\,l},\,g}\\
  \inbr{\inbr{a,\,l},\,g} &[\primi{global}{x}\gets v] & = & \inbr{\inbr{a,\,l},\,g[x\gets v]}
  \end{array}
\]

Now we need to specify the operational semantics for the stack machine (see Fig.~\ref{sm_basic}~-- Fig.~\ref{sm_fun}).
The primitive $\primi{createLocal}{}$ is defined as follows:

\[
\primi{createLocal}{s\;n_a\;n_l}=\inbr{s[n_a...],\,\inbr{[i\in[0..n_a-1]\mapsto s[n_a-i-1]],\,[i\in[0..n_l-1]\mapsto 0]}}\}
\]

\begin{figure}
  \setsubarrow{_{\mathscr SM}}
  \[
    \withenv{P}{\trans{c}{\epsilon}{c}}\ruleno{Stop$_{SM}$}
  \]
  \[
    \trule{\withenv{P}{\trans{\inbr{(x\oplus y)s,\,s_c,\,\sigma,\,\omega}}{p}{c^\prime}}}
          {\withenv{P}{\trans{\inbr{yxs,\,s_c,\,\sigma,\,\omega}}{[\llang{BINOP $\;\otimes$}]p}{c^\prime}}}\ruleno{Binop$_{SM}$}
  \]
  \[
    \trule{\withenv{P}{\trans{\inbr{zs,\,s_c,\,\sigma,\,\omega}}{p}{c^\prime}}}
          {\withenv{P}{\trans{\inbr{s,\,s_c,\,\sigma,\,\omega}}{[\llang{CONST $\;z$}]p}{c^\prime}}}\ruleno{Const$_{SM}$}
  \]
  \[
    \trule{\withenv{P}{\inbr{z,\,\omega^\prime}=\primi{read}{\omega},\,\trans{\inbr{zs,\,s_c,\,\sigma,\,\omega^\prime}}{p}{c^\prime}}}
          {\withenv{P}{\trans{\inbr{s,\,s_c,\,\sigma,\,\omega}}{\llang{READ}\,p}{c^\prime}}}\ruleno{Read$_{SM}$}
  \]
  \[
    \trule{\withenv{P}{\trans{\inbr{s,\,s_c,\,\sigma,\,\primi{write}{z\,\omega}}}{p}{c^\prime}}}
          {\withenv{P}{\trans{\inbr{zs,\,s_c,\,\sigma,\,\omega}}{\llang{WRITE}\,p}{c^\prime}}}\ruleno{Write$_{SM}$}
  \]
  \[ 
    \trule{\withenv{P}{\trans{\inbr{s,\,s_c,\,\sigma,\,\omega}}{p}{c^\prime}}}
          {\withenv{P}{\trans{\inbr{xs,\,s_c,\,\sigma,\,\omega}}{[\llang{DROP}]p}{c^\prime}}}\ruleno{Drop$_{SM}$}
  \]
  \caption{Stack machine: basic rules}
  \label{sm_basic}
\end{figure}

\setarrow{\xRightarrow}
\begin{figure}
  \setsubarrow{_{\mathscr SM}}
    \[
    \trule{\withenv{P}{\trans{\inbr{[\sigma\,(x)]s,\,s_c,\,\sigma,\,\omega}}{p}{c^\prime}}}
          {\withenv{P}{\trans{\inbr{s,\,s_c,\,\sigma,\,\omega}}{[\llang{LD $\;x$}]p}{c^\prime}}}\ruleno{LD$_{SM}$}
  \]
  \[
    \trule{\withenv{P}{\trans{\inbr{[\primi{ref}{x}]s,\,s_c,\,\sigma,\,\omega}}{p}{c^\prime}}}
          {\withenv{P}{\trans{\inbr{s,\,s_c,\,\sigma,\,\omega}}{[\llang{LDA $\;x$}]p}{c^\prime}}}\ruleno{LDA$_{SM}$}
  \]
  \[
    \trule{\withenv{P}{\trans{\inbr{vs,\,s_c,\,\sigma[x\gets v],\,\omega}}{p}{c^\prime}}}
          {\withenv{P}{\trans{\inbr{v[\primi{ref}{x}]s,\,s_c,\,\sigma,\,\omega}}{[\llang{STI}]p}{c^\prime}}}\ruleno{STI$_{SM}$}
  \]
  \[
    \trule{\trans{\inbr{zs,\,s_c,\,\sigma[x\gets z],\,\omega}}{p}{c^\prime}}
          {\trans{\inbr{zs,\,s_c,\,\sigma,\,\omega}}{[\llang{ST $\;x$}]p}{c^\prime}}\ruleno{ST$_{SM}$}
  \]          
  \caption{Stack machine: state operations}
  \label{sm_state}  
\end{figure}

\begin{figure}
  \setsubarrow{_{\mathscr SM}}
  \[
    \trule{\withenv{P}{\trans{c}{p}{c^\prime}}}
          {\withenv{P}{\trans{c}{[\llang{LABEL $\;l$}]p}{c^\prime}}}\ruleno{Label$_{SM}$}
  \]
  \[
    \trule{\withenv{P}{\trans{c}{P[l]}{c^\prime}}}
          {\withenv{P}{\trans{c}{[\llang{JMP $\;l$}]p}{c^\prime}}}\ruleno{JMP$_{SM}$}
  \]
  \[
    \trule{z\ne 0,\quad\withenv{P}{\trans{\inbr{s,\,s_c,\,\sigma,\,\omega}}{P[l]}{c^\prime}}}
          {\withenv{P}{\trans{\inbr{zs,\,s_c,\,\sigma,\,\omega}}{[\llang{CJMP$_{nz}$ $\;l$}]p}{c^\prime}}}\ruleno{CJMP$_{nz}^+$$_{SM}$}
  \]
  \[
    \trule{z = 0,\quad\withenv{P}{\trans{\inbr{s,\,s_c,\,\sigma,\,\omega}}{p}{c^\prime}}}
        {\withenv{P}{\trans{\inbr{zs,\,s_c,\,\sigma,\,\omega}}{[\llang{CJMP$_{nz}$ $\;l$}]p}{c^\prime}}}\ruleno{CJMP$_{nz}^-$$_{SM}$}
  \]
  \[
    \trule{z = 0,\quad\withenv{P}{\trans{\inbr{s,\,s_c,\,\sigma,\,\omega}}{P[l]}{c^\prime}}}
          {\withenv{P}{\trans{\inbr{zs,\,s_c,\,\sigma,\,\omega}}{[\llang{CJMP$_z$ $\;l$}]p}{c^\prime}}}\ruleno{CJMP$_{z}^+$$_{SM}$}
  \]
  \[
    \trule{z \ne 0,\quad\withenv{P}{\trans{\inbr{s,\,s_c,\,\sigma,\,\omega}}{p}{c^\prime}}}
          {\withenv{P}{\trans{\inbr{zs,\,s_c,\,\sigma,\,\omega}}{[\llang{CJMP$_z$ $\;l$}]p}{c^\prime}}}\ruleno{CJMP$_{z}^-$$_{SM}$}
  \]
  \caption{Stack machine: control flow instructions}        
  \label{sm_control}
\end{figure}

\begin{figure}
  \setsubarrow{_{\mathscr SM}}
  \[
    \withenv{P}{\trans{\inbr{s,\,\epsilon,\,\sigma,\,\omega}}{[\llang{END}]p}{\inbr{s,\,\epsilon,\,\sigma,\,\omega}}}\ruleno{EndStop$_{SM}$}
  \]
  \[
<<<<<<< HEAD
    \trule{\withenv{P}{\trans{\inbr{s,\,\epsilon,\,\inbr{\sigma_l,\,\sigma},\,\omega}}{q}{c^\prime}}}
=======
    \trule{\withenv{P}{\trans{\inbr{s,\,s_c,\,\inbr{\sigma_l,\,\sigma},\,\omega}}{q}{c^\prime}}}
>>>>>>> 2170e9fe687b638c8d5e899ffb8208534a484e18
          {\withenv{P}{\trans{\inbr{s,\,\inbr{\sigma_l,\,q}s_c,\,\inbr{\_,\,\sigma},\,\omega}}{[\llang{END}]p}{c^\prime}}}\ruleno{End$_{SM}$}
  \]
  \[
    \trule{\inbr{s^\prime,\,\sigma_l}=\primi{createLocal}{s\;n_a\;n_l}\quad\withenv{P}{\trans{\inbr{s^\prime,\,s_c,\,\inbr{\sigma_l,\,\sigma},\,\omega}}{p}{c^\prime}}}
          {\withenv{P}{\trans{\inbr{s,\,s_c,\,\inbr{\_,\,\sigma},\,\omega}}{[\llang{BEGIN}\;\_\;n_a\;n_l]p}{c^\prime}}}\ruleno{Begin$_{SM}$}
  \]
  \[
    \trule{\withenv{P}{\trans{\inbr{s,\,\inbr{\sigma_l,\,p}s_c,\,\inbr{\sigma_l,\,\sigma},\,\omega}}{P[f]}{c^\prime}}}
          {\withenv{P}{\trans{\inbr{s,\,s_c,\,\inbr{\sigma_l,\,\sigma},\,\omega}}{[\llang{CALL}\;f\;\_]p}{c^\prime}}}\ruleno{Call$_{SM}$}
  \]
  \caption{Stack machine: functions, call, return}
  \label{sm_fun}
\end{figure}

\end{document}
