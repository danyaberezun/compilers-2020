\documentclass{article}

\usepackage{amssymb, amsmath}
\usepackage{alltt}
\usepackage{pslatex}
\usepackage{epigraph}
\usepackage{verbatim}
\usepackage{latexsym}
\usepackage{array}
\usepackage{comment}
\usepackage{makeidx}
\usepackage{listings}
\usepackage{indentfirst}
\usepackage{verbatim}
\usepackage{color}
\usepackage{url}
\usepackage{xspace}
\usepackage{hyperref}
\usepackage{stmaryrd}
\usepackage{amsmath, amsthm, amssymb}
\usepackage{graphicx}
\usepackage{euscript}
\usepackage{mathtools}
\usepackage{mathrsfs}
\usepackage{multirow,bigdelim}
\usepackage{subcaption}
\usepackage{placeins}

\makeatletter

\makeatother

\definecolor{shadecolor}{gray}{1.00}
\definecolor{darkgray}{gray}{0.30}

\def\transarrow{\xrightarrow}
\newcommand{\setarrow}[1]{\def\transarrow{#1}}

\def\padding{\phantom{X}}
\newcommand{\setpadding}[1]{\def\padding{#1}}

\def\subarrow{}
\newcommand{\setsubarrow}[1]{\def\subarrow{#1}}

\newcommand{\trule}[2]{\frac{#1}{#2}}
\newcommand{\crule}[3]{\frac{#1}{#2},\;{#3}}
\newcommand{\withenv}[2]{{#1}\vdash{#2}}
\newcommand{\trans}[3]{{#1}\transarrow{\padding{\textstyle #2}\padding}\subarrow{#3}}
\newcommand{\ctrans}[4]{{#1}\transarrow{\padding#2\padding}\subarrow{#3},\;{#4}}
\newcommand{\llang}[1]{\mbox{\lstinline[mathescape]|#1|}}
\newcommand{\pair}[2]{\inbr{{#1}\mid{#2}}}
\newcommand{\inbr}[1]{\left<{#1}\right>}
\newcommand{\highlight}[1]{\color{red}{#1}}
\newcommand{\ruleno}[1]{\eqno[\scriptsize\textsc{#1}]}
\newcommand{\rulename}[1]{\textsc{#1}}
\newcommand{\inmath}[1]{\mbox{$#1$}}
\newcommand{\lfp}[1]{fix_{#1}}
\newcommand{\gfp}[1]{Fix_{#1}}
\newcommand{\vsep}{\vspace{-2mm}}
\newcommand{\supp}[1]{\scriptsize{#1}}
\newcommand{\sembr}[1]{\llbracket{#1}\rrbracket}
\newcommand{\cd}[1]{\texttt{#1}}
\newcommand{\free}[1]{\boxed{#1}}
\newcommand{\binds}{\;\mapsto\;}
\newcommand{\dbi}[1]{\mbox{\bf{#1}}}
\newcommand{\sv}[1]{\mbox{\textbf{#1}}}
\newcommand{\bnd}[2]{{#1}\mkern-9mu\binds\mkern-9mu{#2}}
\newtheorem{lemma}{Lemma}
\newtheorem{theorem}{Theorem}
\newcommand{\meta}[1]{{\mathcal{#1}}}
\renewcommand{\emptyset}{\varnothing}
\newcommand{\dom}[1]{\mathtt{dom}\;{#1}}
\newcommand{\primi}[2]{\mathbf{#1}\;{#2}}

\definecolor{light-gray}{gray}{0.90}
\newcommand{\graybox}[1]{\colorbox{light-gray}{#1}}

\lstdefinelanguage{lama}{
keywords={read, write, skip,if,then,else,elif,fi,while,do,od,repeat,until,for,fun,local,public,return,import,length,
string,case,of,esac,when,boxed,unboxed,string,sexp,array,infix,infixl,infixr,at,before,after,true,false,eta,lazy},
sensitive=true,
basicstyle=\small,
%commentstyle=\scriptsize\rmfamily,
keywordstyle=\ttfamily\bfseries,
identifierstyle=\ttfamily,
basewidth={0.5em,0.5em},
columns=fixed,
fontadjust=true,
literate={->}{{$\to$}}3,
morecomment=[s][\ttfamily]{(*}{*)},
morecomment=[l][\ttfamily]{--}
}

\lstset{
mathescape=true,
basicstyle=\small,
identifierstyle=\ttfamily,
keywordstyle=\bfseries,
commentstyle=\scriptsize\rmfamily,
basewidth={0.5em,0.5em},
fontadjust=true,
escapechar=!,
language=lama
}

\sloppy

\newcommand{\lama}{$\lambda\kern -.1667em\lower -.5ex\hbox{$a$}\kern -.1000em\lower .2ex\hbox{$\mathcal M$}\kern -.1000em\lower -.5ex\hbox{$a$}$\xspace}

\theoremstyle{definition}

\author{Dmitry Boulytchev}

\begin{document}

\renewcommand{\arraystretch}{1.5}

\section{Statements}

More interesting language~--- a language of simple statements:

\[
\renewcommand{\arraystretch}{1}
\begin{array}{rcl}  
  \mathscr S & = & \mbox{\lstinline|skip|} \\
             &   & \mathscr X \mbox{\lstinline|:=|} \;\mathscr E \\
             &   & \mbox{\lstinline|read (|} \mathscr X \mbox{\lstinline|)|} \\
             &   & \mbox{\lstinline|write (|} \mathscr E \mbox{\lstinline|)|} \\
             &   & \mathscr S \mbox{\lstinline|;|} \mathscr S
\end{array}
\]

Here $\mathscr E, \mathscr X$ stand for the sets of expressions and variables, as in the previous lecture.
Informally, the language allows to write a striaght-line programs which transform an input stream of integers into
output stream of integers.

Again, we define the semantics for this language 

\[
\sembr{\bullet}_{\mathscr S} : \mathscr S \mapsto \mathbb Z^* \to \mathbb Z^*
\]

with the semantic domain of partial functions from integer streams to integer streams. Again, we will
use \emph{big-step operational semantics}: we define a ternary relation ``$\Rightarrow_{\mathscr S}$''

\[
\Rightarrow_{\mathscr S} \subseteq \mathscr C \times \mathscr S \times \mathscr C
\]

where $\mathscr C$~--- a set of possible \emph{configurations} during a program execution.
We will write $c_1\xRightarrow{S}_{\mathscr S}c_2$ instead of $(c_1, S, c_2)\in\Rightarrow_{\mathscr S}$ and informally
interpret the former as ``the execution of a statement $S$ in a configuration $c_1$ completes with the configuration
$c_2$''.

The set of all configuration is defined as

\[
\begin{array}{rcl}
  \mathscr C &=& \Sigma \times \mathscr W\\
  \mathscr W &=& \mathbb Z^* \times \mathbb Z^*
\end{array}
\]

where $\mathscr W$~--- a set of \emph{worlds}, each of which escapsulates some input-output stream pair. For
simplicity, we define the following operations for worlds:

\[
\begin{array}{rcl}
  \primi{read}{\inbr{xi,\,o}}    & = & \inbr{x,\,\inbr{i,\,o}}\\
  \primi{write}{x\,\inbr{i,\,o}} & = & \inbr{i,\,ox}\\
  \primi{out}{\inbr{i,\,o}}      & = & o
\end{array}
\]

The relation ``$\Rightarrow_{\mathscr S}$'' is defined by the following deductive system (see Fig.~\ref{bs_stmt}). The first
three rules are \emph{axioms} as they do not have any premises. Note, according to these rules sometimes a program
cannot do a step in a given configuration: a value of an expression can be undefined in a given state in rules
$\rulename{Assign}$ and $\rulename{Write}$, and there can be no input value in rule $\rulename{Read}$. This style of
a semantics description is called big-step operational semantics, since the results of a computation are
immediately observable at the right hand side of ``$\Rightarrow$'' and, thus, the computation is performed in
a single ``big'' step. And, again, this style of a semantic description can be used to easily implement a
reference interpreter.

With the relation ``$\Rightarrow_{\mathscr S}$'' defined we can abbreviate the ``surface'' semantics for the language of statements:

\setarrow{\xRightarrow}
\setsubarrow{_{\mathscr S}}
\[
\trule{\trans{\inbr{\Lambda,\,\inbr{i,\,\epsilon}}}{S}{\inbr{\sigma,\,\omega}}}
      {\sembr{S}_{\mathscr S}i=\primi{out}{\omega}}
\]


\begin{figure}[t]
\arraycolsep=10pt
\[\trans{c}{\llang{skip}}{c}\ruleno{Skip}\]
\[\trans{\inbr{\sigma,\, \omega}}{\llang{x := $\;\;e$}}{\inbr{\sigma\,[x\gets\sembr{e}_{\mathscr E}\;\sigma],\,\omega}}\ruleno{Assign}\]
\[\trule{\inbr{z,\,\omega^\prime}=\primi{read}{\omega}}
        {\trans{\inbr{\sigma,\, \omega}}{\llang{read ($x$)}}{\inbr{\sigma\,[x\gets z],\,\omega^\prime}}}\ruleno{Read}\]
\[\trans{\inbr{\sigma,\, \omega}}{\llang{write ($e$)}}{\inbr{\sigma,\, \primi{write}{(\sembr{e}_{\mathscr E}\;\sigma)\, \omega}}}\ruleno{Write}\]
\[\trule{\begin{array}{cc}
            \trans{c_1}{S_1}{c^\prime} & \trans{c^\prime}{S_2}{c_2}
         \end{array}}
        {\trans{c_1}{S_1\llang{;}S_2}{c_2}}\ruleno{Seq}\]
\caption{Big-step operational semantics for statements}
\label{bs_stmt}
\end{figure}

\section{Stack Machine}

Stack machine is a simple abstract computational device, which can be used as a convenient model to constructively describe
the compilation process.

In short, stack machine operates on the same configurations, as the language of statements, plus a stack of integers. The
computation, performed by the stack machine, is controlled by a program, which is described as follows:

\[
\renewcommand{\arraystretch}{1}
\begin{array}{rcl}
  \mathscr I & = & \llang{BINOP $\;\otimes$} \\
             &   & \llang{CONST $\;\mathbb N$} \\
             &   & \llang{READ} \\
             &   & \llang{WRITE} \\
             &   & \llang{LD $\;\mathscr X$} \\
             &   & \llang{ST $\;\mathscr X$} \\
  \mathscr P & = & \epsilon \\
             &   & \mathscr I\mathscr P
\end{array}
\]

Here the syntax category $\mathscr I$ stands for \emph{instructions}, $\mathscr P$~--- for \emph{programs}; thus, a program is a finite
string of instructions.

The semantics of stack machine program can be described, again, in the form of big-step operational semantics. This time the set of
stack machine configurations is

\[
\mathscr C_{SM} = \mathbb Z^* \times \mathscr C
\]

where the first component is a stack, and the second~--- a configuration as in the semantics of statement language. The rules are shown on Fig.~\ref{bs_sm}; note,
now we have one axiom and six inference rules (one per instruction).

\setsubarrow{_{\mathscr{SM}}}

As for the statement, with the aid of the relation ``$\Rightarrow_{\mathscr{SM}}$'' we can define the surface semantics of stack machine:

\[
\trule{\trans{\inbr{\epsilon,\,\inbr{\Lambda,\,\inbr{i,\,\epsilon}}}}{p}{\inbr{s, \inbr{\sigma,\,\omega}}}}
      {\sembr{p}_{\mathscr{SM}}\;i=\primi{out}{\omega}}
\]

\begin{figure}[t]
  \[\trans{c}{\epsilon}{c}\ruleno{Stop$_{SM}$}\]
  \[\trule{\trans{\inbr{(x\oplus y)s, c}}{p}{c^\prime}}{\trans{\inbr{yxs, c}}{[\llang{BINOP $\;\otimes$}]p}{c^\prime}}\ruleno{Binop$_{SM}$}\]
  \[\trule{\trans{\inbr{zs, c}}{p}{c^\prime}}{\trans{\inbr{s, c}}{[\llang{CONST $\;z$}]p}{c^\prime}}\ruleno{Const$_{SM}$}\]
  \[\trule{\inbr{z,\,\omega^\prime}=\primi{read}{\omega},\,\trans{\inbr{zs, \inbr{\sigma,\,\omega^\prime}}}{p}{c^\prime}}{\trans{\inbr{s, \inbr{\sigma,\,\omega}}}{\llang{READ}\,p}{c^\prime}}\ruleno{Read$_{SM}$}\]
  \[\trule{\trans{\inbr{s,\,\inbr{\sigma,\,\primi{write}{z\,\omega}}}}{p}{c^\prime}}{\trans{\inbr{zs,\, \inbr{\sigma,\,\omega}}}{\llang{WRITE}\,p}{c^\prime}}\ruleno{Write$_{SM}$}\]
  \[\trule{\trans{\inbr{(\sigma\;x)s,\, \inbr{\sigma,\,\omega}}}{p}{c^\prime}}{\trans{\inbr{s,\, \inbr{\sigma,\,\omega}}}{[\llang{LD $\;x$}]p}{c^\prime}}\ruleno{LD$_{SM}$}\]
  \[\trule{\trans{\inbr{s,\, \inbr{\sigma[x\gets z],\,\omega}}}{p}{c^\prime}}{\trans{\inbr{zs,\, \inbr{\sigma,\,\omega}}}{[\llang{ST $\;x$}]p}{c^\prime}}\ruleno{ST$_{SM}$}\]
  \caption{Big-step operational semantics for stack machine}
  \label{bs_sm}
\end{figure}

\section{A Compiler for the Stack Machine}

A compiler of the statement language into the stack machine is a total mapping

\[
\sembr{\bullet}^{comp} : \mathscr S \mapsto \mathscr P
\]

We can describe the compiler in the form of denotational semantics for the source language. In fact, we can treat the compiler as a \emph{static} semantics, which
maps each program into its stack machine equivalent.

As the source language consists of two syntactic categories (expressions and statments), the compiler has to be ``bootstrapped'' from the compiler for expressions
$\sembr{\bullet}_{\mathscr E}^{comp}$:

\[
\begin{array}{rcl}
  \sembr{x}_{\mathscr E}^{comp}&=&\llang{[LD $\;x$]}\\
  \sembr{n}_{\mathscr E}^{comp}&=&\llang{[CONST $\;n$]}\\
  \sembr{A\otimes B}_{\mathscr E}^{comp}&=&\sembr{A}_{\mathscr E}^{comp}\sembr{B}_{\mathscr E}^{comp}[\llang{BINOP $\;\otimes$}]
\end{array}
\]

And now the main dish:

\[
\begin{array}{rcl}
  \sembr{\llang{$x$ := $e$}}^{comp}&=&\sembr{e}_{\mathscr E}^{comp}\llang{[ST $\;x$]}\\
  \sembr{\llang{read ($x$)}}^{comp}&=&\llang{[READ][ST $\;x$]}\\
  \sembr{\llang{write ($e$)}}^{comp}&=&\sembr{e}_{\mathscr E}^{comp}\llang{[WRITE]}\\
  \sembr{\llang{$S_1$;$\;S_2$}}^{comp}&=&\sembr{S_1}^{comp}\sembr{S_2}^{comp}
\end{array}
\]

\end{document}
