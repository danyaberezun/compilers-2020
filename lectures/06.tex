\documentclass{article}

\usepackage{amssymb, amsmath}
\usepackage{alltt}
\usepackage{pslatex}
\usepackage{epigraph}
\usepackage{verbatim}
\usepackage{latexsym}
\usepackage{array}
\usepackage{comment}
\usepackage{makeidx}
\usepackage{listings}
\usepackage{indentfirst}
\usepackage{verbatim}
\usepackage{color}
\usepackage{url}
\usepackage{xspace}
\usepackage{hyperref}
\usepackage{stmaryrd}
\usepackage{amsmath, amsthm, amssymb}
\usepackage{graphicx}
\usepackage{euscript}
\usepackage{mathtools}
\usepackage{mathrsfs}
\usepackage{multirow,bigdelim}
\usepackage{subcaption}
\usepackage{placeins}

\makeatletter

\makeatother

\definecolor{shadecolor}{gray}{1.00}
\definecolor{darkgray}{gray}{0.30}

\def\transarrow{\xrightarrow}
\newcommand{\setarrow}[1]{\def\transarrow{#1}}

\def\padding{\phantom{X}}
\newcommand{\setpadding}[1]{\def\padding{#1}}

\def\subarrow{}
\newcommand{\setsubarrow}[1]{\def\subarrow{#1}}

\newcommand{\trule}[2]{\dfrac{#1}{#2}}
\newcommand{\crule}[3]{\frac{#1}{#2},\;{#3}}
\newcommand{\withenv}[2]{{#1}\vdash{#2}}
\newcommand{\trans}[3]{{#1}\transarrow{\padding{\textstyle #2}\padding}\subarrow{#3}}
\newcommand{\ctrans}[4]{{#1}\transarrow{\padding#2\padding}\subarrow{#3},\;{#4}}
\newcommand{\llang}[1]{\mbox{\lstinline[mathescape]|#1|}}
\newcommand{\pair}[2]{\inbr{{#1}\mid{#2}}}
\newcommand{\inbr}[1]{\left<{#1}\right>}
\newcommand{\highlight}[1]{\color{red}{#1}}
\newcommand{\ruleno}[1]{\eqno[\scriptsize\textsc{#1}]}
\newcommand{\rulename}[1]{\textsc{#1}}
\newcommand{\inmath}[1]{\mbox{$#1$}}
\newcommand{\lfp}[1]{fix_{#1}}
\newcommand{\gfp}[1]{Fix_{#1}}
\newcommand{\vsep}{\vspace{-2mm}}
\newcommand{\supp}[1]{\scriptsize{#1}}
\newcommand{\sembr}[1]{\llbracket{#1}\rrbracket}
\newcommand{\cd}[1]{\texttt{#1}}
\newcommand{\free}[1]{\boxed{#1}}
\newcommand{\binds}{\;\mapsto\;}
\newcommand{\dbi}[1]{\mbox{\bf{#1}}}
\newcommand{\sv}[1]{\mbox{\textbf{#1}}}
\newcommand{\bnd}[2]{{#1}\mkern-9mu\binds\mkern-9mu{#2}}
\newtheorem{lemma}{Lemma}
\newtheorem{theorem}{Theorem}
\newcommand{\meta}[1]{{\mathcal{#1}}}
\renewcommand{\emptyset}{\varnothing}
\newcommand{\dom}[1]{\mathtt{dom}\;{#1}}
\newcommand{\primi}[2]{\mathbf{#1}\;{#2}}

\definecolor{light-gray}{gray}{0.90}
\newcommand{\graybox}[1]{\colorbox{light-gray}{#1}}

\lstdefinelanguage{lama}{
keywords={read, write, skip,if,then,else,elif,fi,while,do,od,repeat,until,for,fun,local,public,return,import,length,
string,case,of,esac,when,boxed,unboxed,string,sexp,array,infix,infixl,infixr,at,before,after,true,false,eta,lazy,ignore, ref},
sensitive=true,
basicstyle=\small,
%commentstyle=\scriptsize\rmfamily,
keywordstyle=\ttfamily\bfseries,
identifierstyle=\ttfamily,
basewidth={0.5em,0.5em},
columns=fixed,
fontadjust=true,
literate={->}{{$\to$}}3,
morecomment=[s][\ttfamily]{(*}{*)},
morecomment=[l][\ttfamily]{--}
}

\lstset{
mathescape=true,
basicstyle=\small,
identifierstyle=\ttfamily,
keywordstyle=\bfseries,
commentstyle=\scriptsize\rmfamily,
basewidth={0.5em,0.5em},
fontadjust=true,
escapechar=!,
language=lama
}

\sloppy

\newcommand{\lama}{$\lambda\kern -.1667em\lower -.5ex\hbox{$a$}\kern -.1000em\lower .2ex\hbox{$\mathcal M$}\kern -.1000em\lower -.5ex\hbox{$a$}$\xspace}

\theoremstyle{definition}

\author{Dmitry Boulytchev}

\begin{document}

\section{Functions and Local Scopes}

We extend the language with a new category~--- \emph{declarations} ($\mathscr D$), which consists of local variable declarations and
function declarations. At the expression level we add \emph{scope expressions} ($\mathscr S$), nested scopes and function calls. Finally,
the category of programs $\mathscr P$ is scope expression:

\[
\begin{array}{rcll}
  \mathscr S & = & \mathscr D^*\quad \mathscr E & \mbox{--- scope expression}\\
  \mathscr E & += & \lstinline|{ $\mathscr S\;$ }| & \mbox{--- nested scope}\\
             &    & {\mathscr X}\quad\lstinline|($\;\mathscr E^*\;$)| & \mbox{--- function call}\\
  \mathscr D & =  & \mbox{\lstinline|local $\;\mathscr X$|} & \mbox{--- local variable definition} \\
             &    & \mbox{\lstinline|fun $\;\mathscr X\;$ ($\mathscr X^*$) \{$\;\mathscr S\;$\}|} & \mbox{--- function definition}\\
  \mathscr P & =  & \mathscr S & \mbox{--- program}
\end{array}
\]

\subsection{Concrete Syntax}

On the concrete syntax level we stipulate the following conventions:

\begin{enumerate}
\item the expression in scope expression is optional; if no expression is explicitly specified then "\lstinline|skip|" is assumed;
\item local variable definition has to be terminated by a semicolon, for example

  \begin{lstlisting}
     local x;
  \end{lstlisting}
  
\item multiple variable names can be specified in a single definition, for example

  \begin{lstlisting}
     local x, y, z;
  \end{lstlisting}

  is equivalent to
  
  \begin{lstlisting}
     local x;
     local y;
     local z;
  \end{lstlisting}

\item an optional initializer can be specified for a local variable definition; the initializers are reified into sequential assignments,
  preserving their order, for example

  \begin{lstlisting}
     local x = 3;
     local y = x + 5, z = x + y;
  \end{lstlisting}

  is equivalent to

  \begin{lstlisting}
     local x;
     local y;
     local z;

     x := 3;
     y := x + 5;
     z := x + y
  \end{lstlisting}

\item scope expressions are implicitly assumed in the bodies of loops and branches of conditional expressions;

\item in "\lstinline|repeat|" expression the scope of body's \emph{immediate} definitions is implicitly extended to enclose the
  whole expression, thus

  \begin{lstlisting}
     repeat local i; read (i) until x > 0
  \end{lstlisting}

  is equivalent to

  \begin{lstlisting}
     local i;
     repeat read (i) until x > 0
  \end{lstlisting}

\item an implicit scope expression is assumed in the initialization part of "\lstinline|for|"-loop; the scope of its immediate definitions
  is extended to the whole expression as well, thus

  \begin{lstlisting}
     for local i; i := 0, i < 10, i := i+1 do write (i) od
  \end{lstlisting}

  is equavalent to

  \begin{lstlisting}
     local i; for i := 0, i < 10, i := i+1 do write (i) od
  \end{lstlisting}

\end{enumerate}

\subsection{Well-formedness}

\renewcommand{\Ref}{\primi{Ref}{}}
\newcommand{\Val}{\primi{Val}{}}
\newcommand{\Void}{\primi{Void}{}}
\newcommand{\Weak}{\primi{Weak}{}}

We assume functions to always return a values; thus, we need a way to materialize a void into some default value "$\bot$".  We
do this by introducing a new type of attribute~--- "$\Weak$"~--- and adding the following set of rules to
the inference system we used to ensure/restore expression well-formedness (see Fig.~\ref{well_formed_old}).

\begin{figure}[h]
\renewcommand{\arraystretch}{2}
\[
  \begin{array}{cc}
    \withenv{\Weak}{x},\quad x \in \mathscr X       & \withenv{\Weak}{z},\quad z \in \mathbb N \\
  \trule{\withenv{\Val}{l},\quad\withenv{\Val}{r}} 
        {\withenv{\Weak}{l\oplus r}}               & \withenv{\Weak}{\llang{skip; $\;\bot$}} \\[2mm]
  \trule{\withenv{\Ref}{l},\quad\withenv{\Val}{r}}
        {\withenv{\Weak}{\llang{$l\;$ := $\; r$}}} & \withenv{\Weak}{\llang{read ($x$); $\;\bot$}} \\[2mm]
  \trule{\withenv{\Val}{e}}{\withenv{\Weak}{\llang{write ($e$); $\;\bot$}}} & \trule{\withenv{\Val}{e},\quad\withenv{\Void}{s}}{\withenv{\Weak}{\llang{while $\;e\;$ do $\;s\;$ od; $\;\bot$}}} \\[2mm]
  \trule{\withenv{\Val}{e},\quad\withenv{\Void}{s}}{\withenv{\Weak}{\llang{(repeat $\;s\;$ until $\;e$); $\;\bot$ }}} & 
  \end{array}
  \]
  \caption{Well-formedness: additional rules for the old kinds of expressions}
  \label{well_formed_old}  
\end{figure}

We also need to specify the inference rules for the \emph{new} kinds of expressions (see Fig.~\ref{well_formed_new}), 
and, finally, the rules for definitions (see Fig.~\ref{well_formed_def}).


\begin{figure}[h]
\renewcommand{\arraystretch}{2}
\[
  \begin{array}{ccc}
  \trule{\withenv{\Val}{e_1},...,\withenv{\Val}{e_k}} 
        {\withenv{\Val}{f (e_1,...,e_k)}} &   \trule{\withenv{\Val}{e_1},...,\withenv{\Val}{e_k}} 
                                                    {\withenv{\Weak}{f (e_1,...,e_k)}} & \trule{\withenv{\Val}{e_1},...,\withenv{\Val}{e_k}} 
                                                    {\withenv{\Void}{\llang{ignore ($f (e_1,...,e_k)$)}}} \\[2mm]
                                         & \trule{\vdash^{\mathscr D^*}\,d,\quad \withenv{a}{e}}
                                                    {\withenv{a}{d\;e}},\quad d\in\mathscr D^* & \\[2mm]
                                         & \trule{\withenv{a}{e}}
                                                    {\withenv{a}{\llang{\{$\;e\;$\}}}} &                                                     
  \end{array}
  \]
  \caption{Well-formedness: additional rules for the new kinds of expressions}
  \label{well_formed_new}  
\end{figure}

\begin{figure}[h]
\renewcommand{\arraystretch}{2}
\[
   \begin{array}{cc}
    \vdash^{\mathscr D^*}{\epsilon} & \trule{\vdash^{\mathscr D}\,d,\quad \vdash^{\mathscr D^*}\,ds}{\vdash^{\mathscr D^*}\,{d\,ds}} \\[2mm]
    \vdash^{\mathscr D}\,\llang{local $\;x$} & \trule{\withenv{\Weak}{e}}
                                                   {\vdash^{\mathscr D}\,\llang{fun $\;f\;$ (...) \{$\;e\;$\}}}
  \end{array}
  \]
  \caption{Well-formedness: additional rules for definitions}
  \label{well_formed_def}  
\end{figure}


The top-level well-formedness condition for the while program $p$ (which is now a scope expression) is

\[
\withenv{\Void}{p}
\]

\subsection{Semantics}

We now present the big-step operational semantics for functions and scopes. First, we introduce a shortcut for
a multiple substitutions into a state: for a lists of variable names $x\in\mathscr X^*$ and values $v\in \mathscr V^*$  of equal lengths
we define

\[
\sigma[x_i\gets v_i] = \sigma[x_1\gets v_1]...[x_k\gets v_k]
\]

Then, we add a new kind of value~--- a functional value:

\[
\mathscr V += \mathscr X^*\binds\mathscr E
\]

The simplest form of semantics for function calls could be as follows: assume we know that $f$ is a function with arguments $a_1,...,a_k$ and
body $b$; then we evaluate its call $f (e_1,...,e_k)$ as follows:

\setarrow{\xRightarrow}
\begin{figure}[h]
  \[
  \trule{{\setsubarrow{_*}\trans{c}{e_1...e_k}{\inbr{\inbr{\sigma^\prime,\,\omega^\prime},\,v}}}\quad\trans{\inbr{\sigma^\prime[a_i\gets v_i],\,\omega^\prime}}{b}{c^{\prime\prime}}}
        {\trans{c}{f (e_1,...,e_k)}{c^{\prime\prime}}}
  \]
\end{figure}

Thus, however, would describe \emph{dynamic binding} for functions, while our goal to have a semantics with \emph{static binding}.

So, we modify the definition of state as follows:

\[
\Sigma = (2^{\mathscr X}\times (\mathscr X\to \mathscr V))^+
\]

Now a state is a non-empty list of \emph{scopes}; in each scope we have a set of scope variables and a local state. The rightmost element of a state
corresponds to the \emph{global} state; all other elements corespond to properly ordered list of enclosing scopes for a given point
in a program.

We need to redefine two primitives for states: those for reading and assigning values variables. For reading:

\[
\left((L,\,s)\,\sigma\right)\,(x) = \left\{\begin{array}{rcl}
                                   s\;x        & , & x\in L \\
                                   \sigma\,(x) & , & x\not\in L
                                 \end{array}\right.
\]

For assigning:

\[
\left((L,\,s)\,\sigma\right)\,[x\gets v] = \left\{\begin{array}{rcl}
                                            (L,\,s\,[x\gets v])\,\sigma & , & x\in L\\                            
                                            (L,\,s)\,\left(\sigma\,[x\gets v]\right) &,& x\not\in L
                                         \end{array}\right.
\]

With these primitives redefined all existing semantic rules can be preserved; however, we need to put additional requirements for some
rules (assignment, reading from a variable, reading from a world) to ensure that we do not deal with functional values.

Now we need some new rules describing the semantics of new constructs (definitions and function calls). For scope expressions, the following
``structural'' rules are sufficient:

\[
\trule{{\setsubarrow{_{\mathscr D^*}}\trans{\primi{enterScope}{c}}{d}{c^\prime}}\quad\trans{c^\prime}{e}{c^{\prime\prime}}}
      {\trans{c}{d\;e}{\primi{leaveScope}{c^{\prime\prime}}}}\ruleno{Scope}
\]
\[
 \trule{\trans{c}{e}{c^\prime}}
       {\trans{c}{\llang{\{$\;e\;$\}}}{c^\prime}}\ruleno{LocalScope}
\]
\setsubarrow{_{\mathscr D^*}}

We describe the primitives $\primi{enterScope}{}/\;\primi{leaveScope}{}$ below; the transition \hbox{"$\trans{}{}{}$"} is, again, described
with obvious structural rules:

\[
\trans{c}{\epsilon}{c}\ruleno{DefsEmpty}
\]
\[
\trule{{\setsubarrow{_{\mathscr D}}\trans{c}{d}{c^\prime}}\quad\trans{c^\prime}{ds}{c^{\prime\prime}}}
      {\trans{c}{d\,ds}{c^{\prime\prime}}}\ruleno{DefsNonEmpty}
\]

\setsubarrow{_{\mathscr D}}

The interesting part is the relation \hbox{"$\trans{}{}{}$"}:

\[
\trans{c}{\llang{local $\;x$}}{\primi{addName}{x\;0\;c}}\ruleno{DefVar}
\]
\[
\trans{c}{\llang{fun $\;f\;$ ($x_1...x_k$) \{ $\;e\;$ \}}}{\primi{addName}{f\;(x_1..x_k \binds e)\;c}}\ruleno{DefFun}
\]

where the primitive "$\primi{addName}{}$" is defined as follows:

\[
  \primi{addName}\;x\;v\;\inbr{(L,\,s)\,\sigma,\,\omega} = \inbr{(L\cup\{x\},\,s[x\gets v])\,\sigma,\,\omega} 
\]

and we introduce the following shortcut for adding multiple names at once:

\[
\primi{addName^*}{\inbr{x_1,\,x_1}...\inbr{x_k,\,v_k}\;c} = \primi{addName}{x_k\;v_k\;(...(\primi{addName}{x_1\;v_1\;c})...)}
\]


Now we define the primitives $\primi{enterScope}{}/\;\primi{leaveScope}{}$:

\[
\begin{array}{rcl}
  \primi{enterScope}{\inbr{\sigma,\,\omega}} & = & \inbr{(\emptyset,\,\Lambda)\,\sigma,\,\omega}\\
  \primi{leaveScope}{\inbr{(L,\,s)\,\sigma,\,\omega}} & = & \inbr{\sigma,\,\omega}
\end{array}
\]

Finally, we need to define the semantics for function calls:

\setsubarrow{}
\[
\trule{\begin{array}{c}
          \sigma\,f = (a\binds e)\\
          {\setsubarrow{_*}\trans{\inbr{\sigma,\,\omega}}{e_1...e_k}{\inbr{c^\prime,\,v}}}\\
          v_i\in\mathbb Z\\          
          \trans{\primi{addName^*}{\inbr{a_1,\,v_1}...\inbr{a_k,\,v_k}\;(\primi{enterFunction}{c^\prime})}}{e}{\inbr{c^{\prime\prime},\,w}} 
       \end{array}
      }
      {\trans{\inbr{\sigma,\,\omega}}{f (e_1...e_k)}{\inbr{\primi{leaveFunction}{c^\prime\;(\primi{global}{c^{\prime\prime}})},\,w}}}\ruleno{Call}
\]

The primitives "$\primi{enterFunction}{}/\;\primi{leaveFunction}{}/\;\primi{global}{}$" are defined as follows:

\[
\begin{array}{rcl}
  \primi{enterFunction}{\inbr{\sigma,\,\omega}}   & = & \inbr{\primi{enterScope}{(\primi{global}{\sigma})},\,\omega}\\
  \primi{leaveFunction}{\inbr{(L^\prime,\,s^\prime)\,\epsilon,\,\omega}\;(L,\,s)}& = & \inbr{(L,\,s)\,\epsilon,\omega}\\
  \primi{leaveFunction}{\inbr{(L^\prime,\,s^\prime)\,\sigma,\,\omega}\;(L,\,s)}& = & \inbr{(L^\prime,\,s^\prime)\,(\primi{leaveFunction}{\sigma\;(L,\,s)}),\,\omega},\;\sigma\ne\epsilon\\
  \primi{global}{(L,\,s)\,\epsilon} & = & (L,\,s)\\
  \primi{global}{(L,\,s)\,\sigma} & = & \primi{global}{\sigma},\;\sigma\ne\epsilon
\end{array}
\]

\end{document}
